\documentclass[]{article}
\usepackage{lmodern}
\usepackage{amssymb,amsmath}
\usepackage{ifxetex,ifluatex}
\usepackage{fixltx2e} % provides \textsubscript
\ifnum 0\ifxetex 1\fi\ifluatex 1\fi=0 % if pdftex
  \usepackage[T1]{fontenc}
  \usepackage[utf8]{inputenc}
\else % if luatex or xelatex
  \ifxetex
    \usepackage{mathspec}
  \else
    \usepackage{fontspec}
  \fi
  \defaultfontfeatures{Ligatures=TeX,Scale=MatchLowercase}
\fi
% use upquote if available, for straight quotes in verbatim environments
\IfFileExists{upquote.sty}{\usepackage{upquote}}{}
% use microtype if available
\IfFileExists{microtype.sty}{%
\usepackage{microtype}
\UseMicrotypeSet[protrusion]{basicmath} % disable protrusion for tt fonts
}{}
\usepackage[margin=1in]{geometry}
\usepackage{hyperref}
\hypersetup{unicode=true,
            pdfborder={0 0 0},
            breaklinks=true}
\urlstyle{same}  % don't use monospace font for urls
\usepackage{color}
\usepackage{fancyvrb}
\newcommand{\VerbBar}{|}
\newcommand{\VERB}{\Verb[commandchars=\\\{\}]}
\DefineVerbatimEnvironment{Highlighting}{Verbatim}{commandchars=\\\{\}}
% Add ',fontsize=\small' for more characters per line
\usepackage{framed}
\definecolor{shadecolor}{RGB}{248,248,248}
\newenvironment{Shaded}{\begin{snugshade}}{\end{snugshade}}
\newcommand{\KeywordTok}[1]{\textcolor[rgb]{0.13,0.29,0.53}{\textbf{#1}}}
\newcommand{\DataTypeTok}[1]{\textcolor[rgb]{0.13,0.29,0.53}{#1}}
\newcommand{\DecValTok}[1]{\textcolor[rgb]{0.00,0.00,0.81}{#1}}
\newcommand{\BaseNTok}[1]{\textcolor[rgb]{0.00,0.00,0.81}{#1}}
\newcommand{\FloatTok}[1]{\textcolor[rgb]{0.00,0.00,0.81}{#1}}
\newcommand{\ConstantTok}[1]{\textcolor[rgb]{0.00,0.00,0.00}{#1}}
\newcommand{\CharTok}[1]{\textcolor[rgb]{0.31,0.60,0.02}{#1}}
\newcommand{\SpecialCharTok}[1]{\textcolor[rgb]{0.00,0.00,0.00}{#1}}
\newcommand{\StringTok}[1]{\textcolor[rgb]{0.31,0.60,0.02}{#1}}
\newcommand{\VerbatimStringTok}[1]{\textcolor[rgb]{0.31,0.60,0.02}{#1}}
\newcommand{\SpecialStringTok}[1]{\textcolor[rgb]{0.31,0.60,0.02}{#1}}
\newcommand{\ImportTok}[1]{#1}
\newcommand{\CommentTok}[1]{\textcolor[rgb]{0.56,0.35,0.01}{\textit{#1}}}
\newcommand{\DocumentationTok}[1]{\textcolor[rgb]{0.56,0.35,0.01}{\textbf{\textit{#1}}}}
\newcommand{\AnnotationTok}[1]{\textcolor[rgb]{0.56,0.35,0.01}{\textbf{\textit{#1}}}}
\newcommand{\CommentVarTok}[1]{\textcolor[rgb]{0.56,0.35,0.01}{\textbf{\textit{#1}}}}
\newcommand{\OtherTok}[1]{\textcolor[rgb]{0.56,0.35,0.01}{#1}}
\newcommand{\FunctionTok}[1]{\textcolor[rgb]{0.00,0.00,0.00}{#1}}
\newcommand{\VariableTok}[1]{\textcolor[rgb]{0.00,0.00,0.00}{#1}}
\newcommand{\ControlFlowTok}[1]{\textcolor[rgb]{0.13,0.29,0.53}{\textbf{#1}}}
\newcommand{\OperatorTok}[1]{\textcolor[rgb]{0.81,0.36,0.00}{\textbf{#1}}}
\newcommand{\BuiltInTok}[1]{#1}
\newcommand{\ExtensionTok}[1]{#1}
\newcommand{\PreprocessorTok}[1]{\textcolor[rgb]{0.56,0.35,0.01}{\textit{#1}}}
\newcommand{\AttributeTok}[1]{\textcolor[rgb]{0.77,0.63,0.00}{#1}}
\newcommand{\RegionMarkerTok}[1]{#1}
\newcommand{\InformationTok}[1]{\textcolor[rgb]{0.56,0.35,0.01}{\textbf{\textit{#1}}}}
\newcommand{\WarningTok}[1]{\textcolor[rgb]{0.56,0.35,0.01}{\textbf{\textit{#1}}}}
\newcommand{\AlertTok}[1]{\textcolor[rgb]{0.94,0.16,0.16}{#1}}
\newcommand{\ErrorTok}[1]{\textcolor[rgb]{0.64,0.00,0.00}{\textbf{#1}}}
\newcommand{\NormalTok}[1]{#1}
\usepackage{graphicx,grffile}
\makeatletter
\def\maxwidth{\ifdim\Gin@nat@width>\linewidth\linewidth\else\Gin@nat@width\fi}
\def\maxheight{\ifdim\Gin@nat@height>\textheight\textheight\else\Gin@nat@height\fi}
\makeatother
% Scale images if necessary, so that they will not overflow the page
% margins by default, and it is still possible to overwrite the defaults
% using explicit options in \includegraphics[width, height, ...]{}
\setkeys{Gin}{width=\maxwidth,height=\maxheight,keepaspectratio}
\IfFileExists{parskip.sty}{%
\usepackage{parskip}
}{% else
\setlength{\parindent}{0pt}
\setlength{\parskip}{6pt plus 2pt minus 1pt}
}
\setlength{\emergencystretch}{3em}  % prevent overfull lines
\providecommand{\tightlist}{%
  \setlength{\itemsep}{0pt}\setlength{\parskip}{0pt}}
\setcounter{secnumdepth}{0}
% Redefines (sub)paragraphs to behave more like sections
\ifx\paragraph\undefined\else
\let\oldparagraph\paragraph
\renewcommand{\paragraph}[1]{\oldparagraph{#1}\mbox{}}
\fi
\ifx\subparagraph\undefined\else
\let\oldsubparagraph\subparagraph
\renewcommand{\subparagraph}[1]{\oldsubparagraph{#1}\mbox{}}
\fi

%%% Use protect on footnotes to avoid problems with footnotes in titles
\let\rmarkdownfootnote\footnote%
\def\footnote{\protect\rmarkdownfootnote}

%%% Change title format to be more compact
\usepackage{titling}

% Create subtitle command for use in maketitle
\newcommand{\subtitle}[1]{
  \posttitle{
    \begin{center}\large#1\end{center}
    }
}

\setlength{\droptitle}{-2em}

  \title{}
    \pretitle{\vspace{\droptitle}}
  \posttitle{}
    \author{}
    \preauthor{}\postauthor{}
    \date{}
    \predate{}\postdate{}
  

\begin{document}

\subsection{Course: Visual Analytics for Policy and
Management}\label{course-visual-analytics-for-policy-and-management}

\subsubsection{Prof.~José Manuel Magallanes,
PhD}\label{prof.jose-manuel-magallanes-phd}

\begin{center}\rule{0.5\linewidth}{\linethickness}\end{center}

\section{Part 2: Visualizing Tabular
data}\label{part-2-visualizing-tabular-data}

\subsubsection{Univariate Case}\label{univariate-case}

\begin{center}\rule{0.5\linewidth}{\linethickness}\end{center}

Contents:

\begin{enumerate}
\def\labelenumi{\arabic{enumi}.}
\item
  \protect\hyperlink{part1}{Intro.}
\item
  \protect\hyperlink{part2}{Data Types.}
\item
  \protect\hyperlink{part3}{Data Processes.}

  3.1 \protect\hyperlink{part31}{Classification.}

  \begin{itemize}
  \tightlist
  \item
    \protect\hyperlink{part311}{Nominal Data.}
  \item
    \protect\hyperlink{part312}{Ordinal Data.}
  \end{itemize}

  3.2 \protect\hyperlink{part32}{Counting.}

  3.3 \protect\hyperlink{part33}{Measurement.}
\end{enumerate}

\begin{center}\rule{0.5\linewidth}{\linethickness}\end{center}

Most data are commonly organized in tabular format, that is, tables.
When data is in tabular format, cases are organized in rows, while
variables (information about the cases) are organized in columns. Almost
every data you have used in a spreadsheet follows that structure.

For example, when you visit the \href{https://nces.ed.gov/ccd/}{website}
of the Common Core of Data from the US Department of Education, you can
get a data set with detailed information on public schools at the state
of Washington. Let me get a data table I have based on that:

\begin{Shaded}
\begin{Highlighting}[]
\NormalTok{link=}\StringTok{'https://github.com/EvansDataScience/VisualAnalytics_2_tabularData/raw/master/data/eduwa.rda'}

\CommentTok{#getting the data TABLE from the file in the cloud:}
\KeywordTok{load}\NormalTok{(}\DataTypeTok{file=}\KeywordTok{url}\NormalTok{(link))}
\end{Highlighting}
\end{Shaded}

When you are in RStudio, you can view the data table by clicking on its
name at the \emph{Environment} .

It also good to know \emph{how much} info you have:

\begin{Shaded}
\begin{Highlighting}[]
\CommentTok{#number of rows and columns}
\KeywordTok{dim}\NormalTok{(eduwa) }\CommentTok{#nrow(eduwa) ncol(eduwa)}
\end{Highlighting}
\end{Shaded}

This is the list of the 24 columns:

\begin{Shaded}
\begin{Highlighting}[]
\KeywordTok{names}\NormalTok{(eduwa)}
\end{Highlighting}
\end{Shaded}

When dealing with tabular data, you can suspect that you can produce a
visualization for each column, and then for a couple of them
simultaneously, and then for three or more.

In this material, we will pay attention to the \emph{univariate} case;
which is common for searching problems or veryfing outcomes; not for
giving explanations. Then, when dealing with univariate data, you need
to be aware of two things: \emph{what question you are trying to
answer}; and \emph{how to treat a particular variable} to build the plot
that will answer that question.

\protect\hyperlink{part1}{Go to table of contents.}

\subsection{Data Types}\label{data-types}

I can not anticipate all the questions you can try to answer via plots;
but I can tell you that you are always limited by the \textbf{nature} of
the variables you have at hand. Generally speaking, you have either
\emph{categorical} or \emph{numerical} data in each column, and whatever
question you have, you first need to know how that variable you are
planing to use has been encoded, so you can plan the treatment. In R, we
can know that like this:

\begin{Shaded}
\begin{Highlighting}[]
\CommentTok{# this 'width = 70,strict.width='cut' means}
\CommentTok{# you do not want to see more than 70 characters per row.}

\KeywordTok{str}\NormalTok{(eduwa,}\DataTypeTok{width =} \DecValTok{70}\NormalTok{,}\DataTypeTok{strict.width=}\StringTok{'cut'}\NormalTok{)}
\end{Highlighting}
\end{Shaded}

The ones that say \textbf{num} are obviously numbers (numbers in R are
\emph{numeric} when decimal values are detected, and \emph{integer} if
they are not). The ones that say \textbf{chr} are strings, which are
\emph{candidates} to be \textbf{key} columns, which are not variables
themselves, but identifiers of the cases. In this case, the first four
are identifiers, as well as the the 7th, 10th and 15th columns
(\emph{school names}, \emph{address} and \emph{phone}, respectively).
Those variables are not to be analyzed statistically, but may be used
for annotating (7th and 15th column) or for geocoding (10th column).
Notice that for these data, \emph{State} is not to be analyzed as it is
a constant (all rows are from WA); but it would be if the data were from
the whole USA. Then, you see several variables identified as
\emph{factor} or \emph{ordered factor}, which are categorical variables:
they can be analyzed statistically but not in the same way that numbers.

\protect\hyperlink{part1}{Go to table of contents.}

\subsection{Data Processes}\label{data-processes}

Data is obtained via different processes. When you observe reality, you
can \emph{classsify}, \emph{count} or \emph{measure}. Each of these
decisions produces data with some basic characteristics; which are
represented via categories or numerical values.

\subsubsection{Classification}\label{classification}

Categorical data are the output of the classification process. The
classification can propose an incremental or non-incremental
differentiation. The former are named ordinal data and the latter
nominal data. A nominal classification related to education can be type
of school funding: public or private; while an ordinal one can be:
elementary, middle, high, college and graduate school level.

\paragraph{\texorpdfstring{{ 1. Visualization for nominal
scales}}{ 1. Visualization for nominal scales}}\label{visualization-for-nominal-scales}

Let's see some raw values in the variable \emph{\textbf{LocaleType}}:

\begin{Shaded}
\begin{Highlighting}[]
\KeywordTok{head}\NormalTok{(eduwa}\OperatorTok{$}\NormalTok{LocaleType,}\DecValTok{50}\NormalTok{) }\CommentTok{#first fifty values}
\end{Highlighting}
\end{Shaded}

You can not get a clear idea of what a data table has, so a simple
\textbf{frequency table} is the first tool to see what these nominal
data are telling us:

\begin{Shaded}
\begin{Highlighting}[]
\CommentTok{# absolute values}
\KeywordTok{table}\NormalTok{(eduwa}\OperatorTok{$}\NormalTok{LocaleType,}\DataTypeTok{exclude =} \StringTok{'nothing'}\NormalTok{)}
\end{Highlighting}
\end{Shaded}

\begin{Shaded}
\begin{Highlighting}[]
\CommentTok{# relative values}
\NormalTok{absoluteT=}\KeywordTok{table}\NormalTok{(eduwa}\OperatorTok{$}\NormalTok{LocaleType,}\DataTypeTok{exclude =} \StringTok{'nothing'}\NormalTok{)}
\KeywordTok{prop.table}\NormalTok{(absoluteT)}
\end{Highlighting}
\end{Shaded}

This table tells us the location of the public schools. What is the
right visual for this? Sometimes the answer seems obvious, as tradition
or habits give so much weight to decisions. Let's use the very well
known \textbf{pie chart}:

\begin{Shaded}
\begin{Highlighting}[]
\CommentTok{# the pie plots the table:}
\NormalTok{ToPlot=}\KeywordTok{prop.table}\NormalTok{(absoluteT)}
\KeywordTok{pie}\NormalTok{(ToPlot)}
\end{Highlighting}
\end{Shaded}

You should always keep it simple. Then decorate. For example, you can
start improving the plot you already \emph{have}:

\begin{itemize}
\tightlist
\item
  The purple sector does not show a label:
\end{itemize}

\begin{Shaded}
\begin{Highlighting}[]
\KeywordTok{names}\NormalTok{(ToPlot)}
\end{Highlighting}
\end{Shaded}

We could alter the fifth label:

\begin{Shaded}
\begin{Highlighting}[]
\KeywordTok{names}\NormalTok{(ToPlot)[}\DecValTok{5}\NormalTok{]=}\StringTok{'Unknown'}
\end{Highlighting}
\end{Shaded}

\begin{itemize}
\tightlist
\item
  Our plot did not have a title. Titles (and subtitles) are important.
  To give a title, it can be a \emph{question} to be answered by the
  plot:
\end{itemize}

\begin{Shaded}
\begin{Highlighting}[]
\CommentTok{# the pie plots the table:}
\NormalTok{titleText=}\StringTok{'Where are Public Schools located in WA in 2019?'}
\NormalTok{sourceText=}\StringTok{'Source: US Department of Education'}

\KeywordTok{pie}\NormalTok{(ToPlot,}
    \DataTypeTok{main=}\NormalTok{titleText,}
    \DataTypeTok{sub=}\NormalTok{sourceText)}
\end{Highlighting}
\end{Shaded}

The title can \emph{guide} the reader to recognise the purpose of your
plot:

\begin{Shaded}
\begin{Highlighting}[]
\CommentTok{# the pie plots the table:}
\NormalTok{titleText2=}\StringTok{'WA still has schools locations unknown }\CharTok{\textbackslash{}n}\StringTok{ (info from 2018)'}

\KeywordTok{pie}\NormalTok{(ToPlot,}
    \DataTypeTok{main=}\NormalTok{titleText2,}
    \DataTypeTok{sub=}\NormalTok{sourceText)}
\end{Highlighting}
\end{Shaded}

The title can also \emph{suggest} the decision:

\begin{Shaded}
\begin{Highlighting}[]
\CommentTok{# the pie plots the table:}
\NormalTok{titleText3=}\StringTok{'WA needs to fully categorize school locations}\CharTok{\textbackslash{}n}\StringTok{(info from 2018)'}

\KeywordTok{pie}\NormalTok{(ToPlot,}
    \DataTypeTok{main=}\NormalTok{titleText3,}
    \DataTypeTok{sub=}\NormalTok{sourceText)}
\end{Highlighting}
\end{Shaded}

Titles are no that easy to produce. You need to rewrite them many times,
until you find a good combination of words that can be read in less than
ten seconds. Is also good to keep in mind that you must never give your
audience a cacophonous version (a \emph{tongue twister}), and neither
should you include adjectives.

A general rule for any plot is to make it reachable for the audience you
are writing to (read
\href{https://www.red-gate.com/simple-talk/sysadmin/general/5-steps-to-making-your-boss-read-your-technical-reports/}{this}
every day).

Let's do more customization:

\begin{itemize}
\tightlist
\item
  You can use the values as labels. If values between {[}0,1{]}
  represent shares, it is better to use a {[}0,100{]} scale (in \%).
\end{itemize}

\begin{Shaded}
\begin{Highlighting}[]
\NormalTok{ToPlot}\OperatorTok{*}\DecValTok{100}
\end{Highlighting}
\end{Shaded}

\begin{itemize}
\tightlist
\item
  You can customize the colors:
\end{itemize}

\begin{Shaded}
\begin{Highlighting}[]
\CommentTok{# details:}
\NormalTok{ToPlot=ToPlot}\OperatorTok{*}\DecValTok{100} \CommentTok{# preparing labels}
\NormalTok{paletteHere=}\KeywordTok{rainbow}\NormalTok{(}\KeywordTok{length}\NormalTok{(ToPlot)) }\CommentTok{# customizing set of colors}

\CommentTok{# plotting}
\KeywordTok{pie}\NormalTok{(}\DataTypeTok{x=}\NormalTok{ToPlot,}\CommentTok{#table}
    \DataTypeTok{col =}\NormalTok{ paletteHere, }
    \DataTypeTok{labels =}\NormalTok{ ToPlot,}
    \DataTypeTok{main=}\NormalTok{titleText,}
    \DataTypeTok{sub=}\NormalTok{sourceText)}
\end{Highlighting}
\end{Shaded}

The labels need better work:

\begin{Shaded}
\begin{Highlighting}[]
\KeywordTok{paste0}\NormalTok{(}\KeywordTok{round}\NormalTok{(ToPlot,}\DecValTok{2}\NormalTok{),}\StringTok{'%'}\NormalTok{)}
\end{Highlighting}
\end{Shaded}

Then,

\begin{Shaded}
\begin{Highlighting}[]
\NormalTok{plotLabels=}\KeywordTok{paste0}\NormalTok{(}\KeywordTok{round}\NormalTok{(ToPlot,}\DecValTok{2}\NormalTok{),}\StringTok{'%'}\NormalTok{) }\CommentTok{# labels for the slices}
\CommentTok{# plotting}
\KeywordTok{pie}\NormalTok{(}\DataTypeTok{x=}\NormalTok{ToPlot,}\CommentTok{#table}
    \DataTypeTok{col =}\NormalTok{ paletteHere, }
    \DataTypeTok{labels =}\NormalTok{ plotLabels,}
    \DataTypeTok{main=}\NormalTok{titleText,}
    \DataTypeTok{sub=}\NormalTok{sourceText)}
\end{Highlighting}
\end{Shaded}

\begin{itemize}
\tightlist
\item
  You may need to use legends, but considering its cluttering effects:
\end{itemize}

\begin{Shaded}
\begin{Highlighting}[]
\CommentTok{# plotting}
\KeywordTok{pie}\NormalTok{(}\DataTypeTok{x=}\NormalTok{ToPlot,}\CommentTok{#table}
    \DataTypeTok{col =}\NormalTok{ paletteHere, }
    \DataTypeTok{labels =}\NormalTok{ plotLabels,}
    \DataTypeTok{main=}\NormalTok{titleText,}
    \DataTypeTok{sub=}\NormalTok{sourceText)}
\CommentTok{#legend}
\KeywordTok{legend}\NormalTok{(}\DataTypeTok{x=}\StringTok{"right"}\NormalTok{, }\CommentTok{#where}
       \DataTypeTok{legend=}\KeywordTok{names}\NormalTok{(ToPlot), }\CommentTok{#text}
       \DataTypeTok{fill =}\NormalTok{ paletteHere) }\CommentTok{#symbols' colors}
\end{Highlighting}
\end{Shaded}

Most legends need customization:

\begin{Shaded}
\begin{Highlighting}[]
\CommentTok{# MANAGING THE LEGEND:}

\KeywordTok{pie}\NormalTok{(}\DataTypeTok{x=}\NormalTok{ToPlot,}\CommentTok{#table}
    \DataTypeTok{col =}\NormalTok{ paletteHere, }
    \DataTypeTok{labels =}\NormalTok{ plotLabels,}
    \DataTypeTok{main=}\NormalTok{titleText,}
    \DataTypeTok{sub=}\NormalTok{sourceText)}
\CommentTok{#legend}
\KeywordTok{legend}\NormalTok{(}\DataTypeTok{x=}\StringTok{"right"}\NormalTok{, }\CommentTok{#where}
       \DataTypeTok{legend=}\KeywordTok{names}\NormalTok{(ToPlot), }\CommentTok{#text}
       \DataTypeTok{fill =}\NormalTok{ paletteHere,}
       \DataTypeTok{bty =} \StringTok{'n'}\NormalTok{, }\CommentTok{# no box}
       \DataTypeTok{cex =} \FloatTok{0.5}  \CommentTok{# shrink}
\NormalTok{       ) }\CommentTok{#symbols' colors}
\end{Highlighting}
\end{Shaded}

Most people tend to use pie charts with categorical data, but this
\textbf{should not} be the default option to visualize classification
(see this
\href{https://medium.com/geckoboard-under-the-hood/think-before-you-pie-chart-and-more-effective-ways-to-visualize-your-data-862ea3456b26}{discussion}).
Following the advice from the
\href{https://www.youtube.com/watch?v=axx_MYb8-SI}{video} in that post,
we should turn our pie into a \textbf{bar chart}. Let me do it with the
same info I used to build the pie:

\begin{Shaded}
\begin{Highlighting}[]
\CommentTok{# barplot plots the table too}

\KeywordTok{barplot}\NormalTok{(ToPlot,}
         \DataTypeTok{col =}\NormalTok{ paletteHere,}
         \DataTypeTok{main=}\NormalTok{titleText,}
         \DataTypeTok{sub=}\NormalTok{sourceText)}
\end{Highlighting}
\end{Shaded}

We saved some space as no legend was needed (the less \textbf{ink} the
better visual). Speaking of saving, we can get rid of the colors (they
were needed to differentiate the slices).

A very important thing to consider is that axes represent some unit of
measurement, so make sure that unit is shown:

\begin{Shaded}
\begin{Highlighting}[]
\NormalTok{paletteHereNew=}\KeywordTok{c}\NormalTok{(}\StringTok{'gray'}\NormalTok{) }\CommentTok{# just one color}
\CommentTok{# plotting}
\KeywordTok{barplot}\NormalTok{(ToPlot,}
     \DataTypeTok{col =}\NormalTok{ paletteHereNew,}
     \DataTypeTok{border=}\OtherTok{NA}\NormalTok{, }\CommentTok{#no border}
     \DataTypeTok{main=}\NormalTok{titleText,}
     \DataTypeTok{sub=}\NormalTok{sourceText,}
     \DataTypeTok{ylim=}\KeywordTok{c}\NormalTok{(}\DecValTok{0}\NormalTok{,}\DecValTok{50}\NormalTok{),}
     \DataTypeTok{ylab =} \StringTok{'(in %)'} \CommentTok{# show unit}
\NormalTok{     )}
\end{Highlighting}
\end{Shaded}

If you consider \textbf{annotating} the plot, you can use the labels we
used before:

\begin{Shaded}
\begin{Highlighting}[]
\CommentTok{# plotting}
\NormalTok{location=}\KeywordTok{barplot}\NormalTok{(ToPlot,}
     \DataTypeTok{col =}\NormalTok{ paletteHereNew,}
     \DataTypeTok{border=}\OtherTok{NA}\NormalTok{,}
     \DataTypeTok{main=}\NormalTok{titleText,}
     \DataTypeTok{sub=}\NormalTok{sourceText,}
     \DataTypeTok{ylim=}\KeywordTok{c}\NormalTok{(}\DecValTok{0}\NormalTok{,}\DecValTok{50}\NormalTok{),}
     \DataTypeTok{ylab =} \StringTok{'(in %)'}\NormalTok{)}

\CommentTok{#annotating}
\KeywordTok{text}\NormalTok{(}\DataTypeTok{x=}\NormalTok{location,}\DataTypeTok{y=}\NormalTok{ToPlot,}\DataTypeTok{labels=}\NormalTok{plotLabels,}
     \DataTypeTok{pos =} \DecValTok{1}\NormalTok{,}\CommentTok{# if pos=3, text will be on top of bar}
     \DataTypeTok{cex =} \FloatTok{0.8}\NormalTok{) }
\end{Highlighting}
\end{Shaded}

You may decide to change the \textbf{orientation} of the plot:

\begin{Shaded}
\begin{Highlighting}[]
\CommentTok{# plotting}
\NormalTok{location=}\KeywordTok{barplot}\NormalTok{(ToPlot,}
     \DataTypeTok{col =}\NormalTok{ paletteHereNew,}
     \DataTypeTok{border=}\OtherTok{NA}\NormalTok{,}
     \DataTypeTok{main=}\NormalTok{titleText,}
     \DataTypeTok{sub=}\NormalTok{sourceText,}
     \DataTypeTok{ylim=}\KeywordTok{c}\NormalTok{(}\DecValTok{0}\NormalTok{,}\DecValTok{50}\NormalTok{),}
     \DataTypeTok{ylab =} \StringTok{'(in %)'}\NormalTok{,}
     \DataTypeTok{horiz =}\NormalTok{ T) }\CommentTok{# ORIENTATION}

\CommentTok{#annotating}
\KeywordTok{text}\NormalTok{(}\DataTypeTok{x=}\NormalTok{location,}\DataTypeTok{y=}\NormalTok{ToPlot,}\DataTypeTok{labels=}\NormalTok{plotLabels,}
     \DataTypeTok{pos =} \DecValTok{1}\NormalTok{) }\CommentTok{# this is the position of the label}
\end{Highlighting}
\end{Shaded}

The problem above is that changing the orientation, changes the axes.
Then, we need to do more work:

\begin{Shaded}
\begin{Highlighting}[]
\NormalTok{location=}\KeywordTok{barplot}\NormalTok{(ToPlot,}
         \DataTypeTok{col =}\NormalTok{ paletteHereNew,}
         \DataTypeTok{border=}\OtherTok{NA}\NormalTok{,}
         \DataTypeTok{main=}\NormalTok{titleText,}
         \DataTypeTok{sub=}\NormalTok{sourceText,}
         \DataTypeTok{xlim=}\KeywordTok{c}\NormalTok{(}\DecValTok{0}\NormalTok{,}\DecValTok{50}\NormalTok{), }\CommentTok{#change to xlim}
         \DataTypeTok{xlab =} \StringTok{'(in %)'}\NormalTok{, }\CommentTok{#change to xlab}
         \DataTypeTok{horiz =}\NormalTok{ T)}

\CommentTok{#annotating}
\KeywordTok{text}\NormalTok{(}\DataTypeTok{x=}\NormalTok{ToPlot,}\DataTypeTok{y=}\NormalTok{location, }\CommentTok{#change of x and y}
     \DataTypeTok{labels=}\NormalTok{plotLabels,}
     \DataTypeTok{pos =} \DecValTok{4}\NormalTok{)  }\CommentTok{# change position of the label}
\end{Highlighting}
\end{Shaded}

A little more work on the categories names:

\begin{Shaded}
\begin{Highlighting}[]
\NormalTok{location=}\KeywordTok{barplot}\NormalTok{(ToPlot,}
         \DataTypeTok{col =}\NormalTok{ paletteHereNew,}
         \DataTypeTok{border=}\OtherTok{NA}\NormalTok{,}
         \DataTypeTok{main=}\NormalTok{titleText,}
         \DataTypeTok{sub=}\NormalTok{sourceText,}
         \DataTypeTok{cex.names =} \FloatTok{0.7}\NormalTok{, }\CommentTok{#shrink category names}
         \DataTypeTok{xlim=}\KeywordTok{c}\NormalTok{(}\DecValTok{0}\NormalTok{,}\DecValTok{50}\NormalTok{), }
         \DataTypeTok{xlab =} \StringTok{'(in %)'}\NormalTok{, }
         \DataTypeTok{horiz =}\NormalTok{ T)}

\CommentTok{#annotating}
\KeywordTok{text}\NormalTok{(}\DataTypeTok{x=}\NormalTok{ToPlot,}\DataTypeTok{y=}\NormalTok{location,}\DataTypeTok{labels=}\NormalTok{plotLabels,}\DataTypeTok{pos =} \DecValTok{4}\NormalTok{)  }
\end{Highlighting}
\end{Shaded}

We made the right changes, but some things do not look well. It would be
better if:

\begin{itemize}
\tightlist
\item
  The subtitle (\emph{source}) and the label of the x-axis were not that
  close. A good step will be to have the subtitle as an element of its
  own, which allows, for instance, to decide its alignment and size:
\end{itemize}

\begin{Shaded}
\begin{Highlighting}[]
\NormalTok{location=}\KeywordTok{barplot}\NormalTok{(ToPlot,}
         \DataTypeTok{col =}\NormalTok{ paletteHereNew,}
         \DataTypeTok{border=}\OtherTok{NA}\NormalTok{,}
         \DataTypeTok{main=}\NormalTok{titleText, }\CommentTok{# no sub here!}
         \DataTypeTok{xlim=}\KeywordTok{c}\NormalTok{(}\DecValTok{0}\NormalTok{,}\DecValTok{50}\NormalTok{), }
         \DataTypeTok{cex.names =} \FloatTok{0.5}\NormalTok{,}
         \DataTypeTok{xlab =} \StringTok{'(in %)'}\NormalTok{, }
         \DataTypeTok{horiz =}\NormalTok{ T)}

\CommentTok{# annotating}
\KeywordTok{text}\NormalTok{(}\DataTypeTok{x=}\NormalTok{ToPlot,}\DataTypeTok{y=}\NormalTok{location,}\DataTypeTok{labels=}\NormalTok{plotLabels,}\DataTypeTok{pos =} \DecValTok{4}\NormalTok{)  }

\CommentTok{# subtitle}
\KeywordTok{title}\NormalTok{(}\DataTypeTok{sub=}\NormalTok{sourceText, }
      \DataTypeTok{adj=}\DecValTok{0}\NormalTok{,}\CommentTok{#adj=1 aligns to rigth.}
      \DataTypeTok{cex.sub=}\FloatTok{0.7}\NormalTok{) }\CommentTok{#shrinking text}
\end{Highlighting}
\end{Shaded}

\begin{itemize}
\tightlist
\item
  To have the label of the x-axis closer to the axis itself, we need to
  alter the \emph{graphical parameters}:
\end{itemize}

\begin{Shaded}
\begin{Highlighting}[]
\CommentTok{# changing parameters}
\CommentTok{# (distanceOfUnit To plot, }
\CommentTok{# distanceOfAxislabels to plot,}
\CommentTok{# distance ofTicks to plot)}
\CommentTok{# default is: mgp=c(3, 1, 0)}

\KeywordTok{par}\NormalTok{(}\DataTypeTok{mgp=}\KeywordTok{c}\NormalTok{(}\FloatTok{0.5}\NormalTok{,}\FloatTok{0.5}\NormalTok{,}\DecValTok{0}\NormalTok{)) }
\NormalTok{#####}

\NormalTok{location=}\KeywordTok{barplot}\NormalTok{(ToPlot,}
         \DataTypeTok{col =}\NormalTok{ paletteHereNew,}
         \DataTypeTok{border=}\OtherTok{NA}\NormalTok{,}
         \DataTypeTok{main=}\NormalTok{titleText,}
         \DataTypeTok{xlim=}\KeywordTok{c}\NormalTok{(}\DecValTok{0}\NormalTok{,}\DecValTok{50}\NormalTok{), }
         \DataTypeTok{xlab =} \StringTok{'(in %)'}\NormalTok{,}
         \DataTypeTok{cex.names =} \FloatTok{0.6}\NormalTok{,}
         \DataTypeTok{cex.lab=}\FloatTok{0.6}\NormalTok{, }\CommentTok{# shrinking label text}
         \DataTypeTok{horiz =}\NormalTok{ T) }

\KeywordTok{text}\NormalTok{(}\DataTypeTok{x=}\NormalTok{ToPlot,}\DataTypeTok{y=}\NormalTok{location,}\DataTypeTok{labels=}\NormalTok{plotLabels,}\DataTypeTok{pos =} \DecValTok{4}\NormalTok{) }

\KeywordTok{title}\NormalTok{(}\DataTypeTok{sub=}\NormalTok{sourceText, }\DataTypeTok{adj=}\DecValTok{0}\NormalTok{,}\DataTypeTok{cex.sub=}\FloatTok{0.7}\NormalTok{,}
      \DataTypeTok{line =} \DecValTok{3}\NormalTok{) }\CommentTok{#push the text down}
\end{Highlighting}
\end{Shaded}

\begin{itemize}
\tightlist
\item
  It is generally a good idea to add a \textbf{reference line}, which
  can represent an expected value or another relevant value. Since I
  have four different locations (not considering the missing ones), let
  me put a line to signal the 25\% (uniform share among four locations):
\end{itemize}

\begin{Shaded}
\begin{Highlighting}[]
\NormalTok{titleText2=}\StringTok{'Are all locations getting a fair share of public schools in WA?'}


\KeywordTok{par}\NormalTok{(}\DataTypeTok{mgp=}\KeywordTok{c}\NormalTok{(}\DecValTok{1}\NormalTok{,}\FloatTok{0.5}\NormalTok{,}\DecValTok{0}\NormalTok{)) }
\NormalTok{location=}\KeywordTok{barplot}\NormalTok{(ToPlot,}
         \DataTypeTok{col =}\NormalTok{ paletteHereNew,}
         \DataTypeTok{border=}\OtherTok{NA}\NormalTok{,}
         \DataTypeTok{main=}\NormalTok{titleText2,}
         \DataTypeTok{xlim=}\KeywordTok{c}\NormalTok{(}\DecValTok{0}\NormalTok{,}\DecValTok{50}\NormalTok{), }
         \DataTypeTok{cex.names =} \FloatTok{0.6}\NormalTok{,}
         \DataTypeTok{cex.lab=}\FloatTok{0.6}\NormalTok{,}
         \DataTypeTok{xlab =} \StringTok{'(in %)'}\NormalTok{,}
         \DataTypeTok{horiz =}\NormalTok{ T}
\NormalTok{         ) }

\KeywordTok{text}\NormalTok{(}\DataTypeTok{x=}\NormalTok{ToPlot,}\DataTypeTok{y=}\NormalTok{location,}\DataTypeTok{labels=}\NormalTok{plotLabels,}\DataTypeTok{pos =} \DecValTok{4}\NormalTok{) }
\KeywordTok{title}\NormalTok{(}\DataTypeTok{sub=}\NormalTok{sourceText, }\DataTypeTok{adj=}\DecValTok{0}\NormalTok{,}\DataTypeTok{cex.sub=}\FloatTok{0.7}\NormalTok{,}
      \DataTypeTok{line =} \DecValTok{3}\NormalTok{) }

\CommentTok{# reference line}
\KeywordTok{abline}\NormalTok{(}\DataTypeTok{v=}\DecValTok{25}\NormalTok{,}\CommentTok{#position vertical}
       \DataTypeTok{lty=}\DecValTok{3}\NormalTok{,}\CommentTok{#type}
       \DataTypeTok{lwd=}\DecValTok{3}\NormalTok{)}\CommentTok{#width}
\end{Highlighting}
\end{Shaded}

Again, adding another element requires adjusting what we had. What about
writing your own axis-values and reducing the bar annotations:

\begin{Shaded}
\begin{Highlighting}[]
\KeywordTok{par}\NormalTok{(}\DataTypeTok{mgp=}\KeywordTok{c}\NormalTok{(}\DecValTok{1}\NormalTok{,}\FloatTok{0.5}\NormalTok{,}\DecValTok{0}\NormalTok{)) }
\NormalTok{location=}\KeywordTok{barplot}\NormalTok{(ToPlot,}
         \DataTypeTok{col =}\NormalTok{ paletteHereNew,}
         \DataTypeTok{border=}\OtherTok{NA}\NormalTok{,}
         \DataTypeTok{main=}\NormalTok{titleText2,}
         \DataTypeTok{xlim=}\KeywordTok{c}\NormalTok{(}\DecValTok{0}\NormalTok{,}\DecValTok{50}\NormalTok{), }
         \DataTypeTok{xlab =} \StringTok{'(in %)'}\NormalTok{,}
         \DataTypeTok{cex.names=}\FloatTok{0.6}\NormalTok{,}
         \DataTypeTok{cex.lab=}\FloatTok{0.6}\NormalTok{,}
         \DataTypeTok{las=}\DecValTok{2}\NormalTok{,}
         \DataTypeTok{horiz =}\NormalTok{ T,}
         \DataTypeTok{xaxt=}\StringTok{"n"}\NormalTok{) }\CommentTok{# no x-axis, so I customize it below...}

\KeywordTok{text}\NormalTok{(}\DataTypeTok{x=}\NormalTok{ToPlot,}\DataTypeTok{y=}\NormalTok{location,}\DataTypeTok{labels=}\NormalTok{plotLabels,}\DataTypeTok{pos =} \DecValTok{4}\NormalTok{,}\DataTypeTok{cex =} \FloatTok{0.7}\NormalTok{) }
\KeywordTok{title}\NormalTok{(}\DataTypeTok{sub=}\NormalTok{sourceText, }\DataTypeTok{adj=}\DecValTok{0}\NormalTok{,}\DataTypeTok{cex.sub=}\FloatTok{0.7}\NormalTok{,}\DataTypeTok{line =} \DecValTok{3}\NormalTok{) }

\CommentTok{#reference line}
\KeywordTok{abline}\NormalTok{(}\DataTypeTok{v=}\DecValTok{25}\NormalTok{,}\DataTypeTok{lty=}\DecValTok{3}\NormalTok{,}\DataTypeTok{lwd=}\DecValTok{3}\NormalTok{)}


\CommentTok{# customizing tick values}
\NormalTok{newXvalues<-}\KeywordTok{c}\NormalTok{(}\DecValTok{0}\NormalTok{,}\DecValTok{10}\NormalTok{,}\DecValTok{25}\NormalTok{,}\DecValTok{40}\NormalTok{,}\DecValTok{50}\NormalTok{) }\CommentTok{# you just want to show this on the axis}
\KeywordTok{axis}\NormalTok{(}\DataTypeTok{side=}\DecValTok{1}\NormalTok{, }
     \DataTypeTok{at=}\NormalTok{newXvalues, }
     \DataTypeTok{labels =}\NormalTok{ newXvalues,}
     \DataTypeTok{cex.axis=}\FloatTok{0.8}\NormalTok{)}
\end{Highlighting}
\end{Shaded}

So far, we have used the \emph{basic} R capabilities for plotting.

There are alternative libraries, like \textbf{ggplot2}, that are also
frequently used. However, it has a different approach, which allows to
add layers that let you customize your plot. The classic approach for
ggplot is:

\begin{itemize}
\tightlist
\item
  Avoid missing values and prepare frequency table. We replaced the
  missing values (now they are `Unknown'). Here, you need to transform
  the \emph{table} into a \textbf{data frame}:
\end{itemize}

\begin{Shaded}
\begin{Highlighting}[]
\NormalTok{tableFreq=}\KeywordTok{as.data.frame}\NormalTok{(ToPlot)}
\KeywordTok{names}\NormalTok{(tableFreq)=}\KeywordTok{c}\NormalTok{(}\StringTok{"locale"}\NormalTok{,}\StringTok{"pct"}\NormalTok{)}
\CommentTok{#you have:}
\NormalTok{tableFreq}
\end{Highlighting}
\end{Shaded}

\begin{itemize}
\tightlist
\item
  Call the library:
\end{itemize}

\begin{Shaded}
\begin{Highlighting}[]
\KeywordTok{library}\NormalTok{(ggplot2)}
\end{Highlighting}
\end{Shaded}

\begin{itemize}
\tightlist
\item
  Create the base \emph{object}, which is not a plot, just informing the
  main variables:
\end{itemize}

\begin{Shaded}
\begin{Highlighting}[]
\CommentTok{#base GGPLOT2 starts with a "base", telling WHAT VARIABLES TO PLOT}
\NormalTok{base=}\StringTok{ }\KeywordTok{ggplot}\NormalTok{(}\DataTypeTok{data =}\NormalTok{ tableFreq, }
             \KeywordTok{aes}\NormalTok{(}\DataTypeTok{x =}\NormalTok{ locale,}
                 \DataTypeTok{y =}\NormalTok{ pct)) }
\end{Highlighting}
\end{Shaded}

\begin{itemize}
\tightlist
\item
  On top of the previous object, add the layer that produces the main
  plots (the next layers will add or customize elements in the plot):
\end{itemize}

\begin{Shaded}
\begin{Highlighting}[]
\NormalTok{plot1 =}\StringTok{ }\NormalTok{base }\OperatorTok{+}\StringTok{ }\KeywordTok{geom_bar}\NormalTok{(}\DataTypeTok{fill =}\StringTok{"gray"}\NormalTok{,}
                        \DataTypeTok{stat =} \StringTok{'identity'}\NormalTok{) }\CommentTok{# y is just what it is!}
\NormalTok{plot1}
\end{Highlighting}
\end{Shaded}

\begin{itemize}
\tightlist
\item
  We can now pay attention to the titles:
\end{itemize}

\begin{Shaded}
\begin{Highlighting}[]
\NormalTok{plot2 =}\StringTok{ }\NormalTok{plot1 }\OperatorTok{+}\StringTok{ }\KeywordTok{labs}\NormalTok{(}\DataTypeTok{title=}\NormalTok{titleText2,}
                     \DataTypeTok{x =}\OtherTok{NULL}\NormalTok{, }
                     \DataTypeTok{y =} \OtherTok{NULL}\NormalTok{,}
                     \DataTypeTok{caption =}\NormalTok{ sourceText)}
\NormalTok{plot2}
\end{Highlighting}
\end{Shaded}

\begin{itemize}
\tightlist
\item
  Add the reference lines:
\end{itemize}

\begin{Shaded}
\begin{Highlighting}[]
\NormalTok{plot3 =}\StringTok{ }\NormalTok{plot2 }\OperatorTok{+}\StringTok{ }\KeywordTok{geom_hline}\NormalTok{(}\DataTypeTok{yintercept =} \DecValTok{25}\NormalTok{, }\CommentTok{#where}
                           \DataTypeTok{linetype=}\StringTok{"dashed"}\NormalTok{, }
                           \DataTypeTok{size=}\FloatTok{1.5}\NormalTok{, }\CommentTok{#thickness}
                           \DataTypeTok{alpha=}\FloatTok{0.5}\NormalTok{) }\CommentTok{#transparency}
\NormalTok{plot3}
\end{Highlighting}
\end{Shaded}

\begin{itemize}
\tightlist
\item
  Customize the axes:
\end{itemize}

\begin{Shaded}
\begin{Highlighting}[]
\KeywordTok{library}\NormalTok{(scales)}

\CommentTok{# customize Y axis}
\NormalTok{plot4 =}\StringTok{ }\NormalTok{plot3 }\OperatorTok{+}\StringTok{ }\KeywordTok{scale_y_continuous}\NormalTok{(}\DataTypeTok{breaks=}\KeywordTok{c}\NormalTok{(}\DecValTok{0}\NormalTok{,}\DecValTok{10}\NormalTok{, }\DecValTok{25}\NormalTok{,}\DecValTok{40}\NormalTok{,}\DecValTok{50}\NormalTok{),}
                                 \DataTypeTok{limits =} \KeywordTok{c}\NormalTok{(}\DecValTok{0}\NormalTok{, }\DecValTok{50}\NormalTok{), }\CommentTok{# expand = c(0, 0),}
                                 \DataTypeTok{labels=}\NormalTok{scales}\OperatorTok{::}\KeywordTok{unit_format}\NormalTok{(}\DataTypeTok{suffix =} \StringTok{'%'}\NormalTok{)) }
\NormalTok{plot4}
\end{Highlighting}
\end{Shaded}

\begin{itemize}
\tightlist
\item
  Less ink and title/subtitle positions:
\end{itemize}

\begin{Shaded}
\begin{Highlighting}[]
\NormalTok{plot5 =}\StringTok{ }\NormalTok{plot4 }\OperatorTok{+}\StringTok{ }\KeywordTok{theme}\NormalTok{(}\DataTypeTok{panel.background =} \KeywordTok{element_rect}\NormalTok{(}\DataTypeTok{fill =} \StringTok{"white"}\NormalTok{,}
                                                    \DataTypeTok{colour =} \StringTok{"grey50"}\NormalTok{),}
                    \DataTypeTok{plot.caption =} \KeywordTok{element_text}\NormalTok{(}\DataTypeTok{hjust =} \DecValTok{0}\NormalTok{), }\CommentTok{# default was 1}
                    \DataTypeTok{plot.title =} \KeywordTok{element_text}\NormalTok{(}\DataTypeTok{hjust =} \FloatTok{0.5}\NormalTok{))}
\NormalTok{plot5}
\end{Highlighting}
\end{Shaded}

\begin{itemize}
\tightlist
\item
  annotating the bars:
\end{itemize}

\begin{Shaded}
\begin{Highlighting}[]
\NormalTok{plot6 =}\StringTok{ }\NormalTok{plot5 }\OperatorTok{+}\StringTok{ }\KeywordTok{geom_text}\NormalTok{(}\KeywordTok{aes}\NormalTok{(}\DataTypeTok{y =}\NormalTok{ pct ,}
                            \DataTypeTok{label =} \KeywordTok{paste0}\NormalTok{(}\KeywordTok{round}\NormalTok{(pct,}\DecValTok{2}\NormalTok{), }\StringTok{'%'}\NormalTok{)),}
                        \DataTypeTok{vjust=}\DecValTok{1}\NormalTok{, }\CommentTok{# if flipping 'hjust'}
                        \DataTypeTok{size =} \DecValTok{3}\NormalTok{)}
\CommentTok{# wanna flip the plot?}
\NormalTok{plot6 }\CommentTok{#+ coord_flip()}
\end{Highlighting}
\end{Shaded}

Bar plots are the default option for categorical variables. In general,
you see the distribution of the classification, which allows you to
identify \textbf{concentration}. For that reason, ordering the bars by
height can be helpful:

\begin{Shaded}
\begin{Highlighting}[]
\NormalTok{###}
\NormalTok{ToPlotOrd=}\KeywordTok{sort}\NormalTok{(ToPlot)}
\NormalTok{###}

\KeywordTok{par}\NormalTok{(}\DataTypeTok{mgp=}\KeywordTok{c}\NormalTok{(}\DecValTok{1}\NormalTok{,}\FloatTok{0.5}\NormalTok{,}\DecValTok{0}\NormalTok{)) }\CommentTok{# distance label, tickText,tick}
\NormalTok{location=}\KeywordTok{barplot}\NormalTok{(ToPlotOrd,}
         \DataTypeTok{col =}\NormalTok{ paletteHereNew,}
         \DataTypeTok{border=}\OtherTok{NA}\NormalTok{,}
         \DataTypeTok{main=}\NormalTok{titleText2,}
         \DataTypeTok{xlim=}\KeywordTok{c}\NormalTok{(}\DecValTok{0}\NormalTok{,}\DecValTok{50}\NormalTok{), }
         \DataTypeTok{xlab =} \StringTok{'(in %)'}\NormalTok{,}
         \DataTypeTok{horiz =}\NormalTok{ T,}
         \DataTypeTok{cex.names =} \FloatTok{0.7}\NormalTok{,}
         \DataTypeTok{cex.lab=}\FloatTok{0.6}\NormalTok{,}
         \DataTypeTok{xaxt=}\StringTok{"n"}\NormalTok{) }\CommentTok{# no x-axis, so I customize it below...}

\KeywordTok{text}\NormalTok{(}\DataTypeTok{x=}\NormalTok{ToPlotOrd,}\DataTypeTok{y=}\NormalTok{location,}\DataTypeTok{labels=}\NormalTok{plotLabels,}\DataTypeTok{pos =} \DecValTok{2}\NormalTok{,}\DataTypeTok{cex =} \FloatTok{0.7}\NormalTok{) }
\KeywordTok{title}\NormalTok{(}\DataTypeTok{sub=}\NormalTok{sourceText, }\DataTypeTok{adj=}\DecValTok{0}\NormalTok{,}\DataTypeTok{cex.sub=}\FloatTok{0.7}\NormalTok{,}\DataTypeTok{line =} \DecValTok{3}\NormalTok{) }

\CommentTok{# reference line}
\KeywordTok{abline}\NormalTok{(}\DataTypeTok{v=}\DecValTok{25}\NormalTok{,}\DataTypeTok{lty=}\DecValTok{3}\NormalTok{,}\DataTypeTok{lwd=}\DecValTok{3}\NormalTok{)}

\CommentTok{# customizong tick values}
\NormalTok{xtick<-}\KeywordTok{c}\NormalTok{(}\DecValTok{0}\NormalTok{,}\DecValTok{10}\NormalTok{,}\DecValTok{25}\NormalTok{,}\DecValTok{40}\NormalTok{,}\DecValTok{50}\NormalTok{)}
\KeywordTok{axis}\NormalTok{(}\DataTypeTok{side=}\DecValTok{1}\NormalTok{, }\DataTypeTok{at=}\NormalTok{xtick, }\DataTypeTok{labels =}\NormalTok{ xtick,}\DataTypeTok{cex.axis=}\FloatTok{0.8}\NormalTok{)}
\end{Highlighting}
\end{Shaded}

The plot above simply change the order of the \emph{table}. If you want
to do the same with \emph{ggplot} you should try the command:

\begin{Shaded}
\begin{Highlighting}[]
\NormalTok{tableFreq[}\KeywordTok{order}\NormalTok{(}\OperatorTok{-}\NormalTok{tableFreq}\OperatorTok{$}\NormalTok{pct),]}
\end{Highlighting}
\end{Shaded}

{ Exercise:Use ggplot to show a bar plot ordered by share size.}

We could use our reference line to show gaps or differences. In this
case, the \textbf{Lollipop} plot may be useful. This one is just a
replacement for a bar plot:

\begin{Shaded}
\begin{Highlighting}[]
\NormalTok{base =}\StringTok{ }\KeywordTok{ggplot}\NormalTok{(tableFreq, }\KeywordTok{aes}\NormalTok{(}\DataTypeTok{x=}\NormalTok{locale,pct)) }
\NormalTok{lolliplot1=base }\OperatorTok{+}\StringTok{ }\KeywordTok{geom_segment}\NormalTok{(}\KeywordTok{aes}\NormalTok{(}\DataTypeTok{y =} \DecValTok{0}\NormalTok{, }
                                   \DataTypeTok{x =}\NormalTok{ locale, }
                                   \DataTypeTok{yend =}\NormalTok{ pct, }
                                   \DataTypeTok{xend =}\NormalTok{ locale), }\DataTypeTok{color =} \StringTok{"grey50"}\NormalTok{) }
\NormalTok{lolliplot1 }\OperatorTok{+}\StringTok{ }\KeywordTok{geom_point}\NormalTok{()}
\end{Highlighting}
\end{Shaded}

And, if you order the data frame:

\begin{Shaded}
\begin{Highlighting}[]
\NormalTok{tableFreq[}\KeywordTok{order}\NormalTok{(tableFreq}\OperatorTok{$}\NormalTok{pct),]}
\end{Highlighting}
\end{Shaded}

You can get:

\begin{Shaded}
\begin{Highlighting}[]
\CommentTok{# reordering DF steps:}
\NormalTok{tableFreqO=tableFreq[}\KeywordTok{order}\NormalTok{(tableFreq}\OperatorTok{$}\NormalTok{pct),]}


\NormalTok{base =}\StringTok{ }\KeywordTok{ggplot}\NormalTok{(tableFreqO, }\KeywordTok{aes}\NormalTok{(locale,pct)) }
\NormalTok{lolliplot1=base }\OperatorTok{+}\StringTok{ }\KeywordTok{geom_segment}\NormalTok{(}\KeywordTok{aes}\NormalTok{(}\DataTypeTok{y =} \DecValTok{0}\NormalTok{, }
                                   \DataTypeTok{x =}\NormalTok{ locale, }
                                   \DataTypeTok{yend =}\NormalTok{ pct, }
                                   \DataTypeTok{xend =}\NormalTok{ locale), }\DataTypeTok{color =} \StringTok{"gray"}\NormalTok{) }
\NormalTok{lolliplot2 =}\StringTok{ }\NormalTok{lolliplot1 }\OperatorTok{+}\StringTok{ }\KeywordTok{geom_point}\NormalTok{()}
\NormalTok{lolliplot2 }\OperatorTok{+}\StringTok{ }\KeywordTok{scale_x_discrete}\NormalTok{(}\DataTypeTok{limits=}\NormalTok{tableFreqO}\OperatorTok{$}\NormalTok{locale) }\CommentTok{# key element}
\end{Highlighting}
\end{Shaded}

And, what about changing the axis values so that we can identify the
gaps:

\begin{Shaded}
\begin{Highlighting}[]
\CommentTok{# new variable}
\NormalTok{tableFreqO}\OperatorTok{$}\NormalTok{gap=tableFreqO}\OperatorTok{$}\NormalTok{pct}\OperatorTok{-}\DecValTok{25}

\CommentTok{# plot the new variable}
\NormalTok{base =}\StringTok{ }\KeywordTok{ggplot}\NormalTok{(tableFreqO, }\KeywordTok{aes}\NormalTok{(locale,gap)) }

\NormalTok{lolliplot1=base }\OperatorTok{+}\StringTok{ }\KeywordTok{geom_segment}\NormalTok{(}\KeywordTok{aes}\NormalTok{(}\DataTypeTok{y =} \DecValTok{0}\NormalTok{, }
                                   \DataTypeTok{x =}\NormalTok{ locale, }
                                   \DataTypeTok{yend =}\NormalTok{ gap, }
                                   \DataTypeTok{xend =}\NormalTok{ locale), }\DataTypeTok{color =} \StringTok{"gray"}\NormalTok{) }
\NormalTok{lolliplot2 =}\StringTok{ }\NormalTok{lolliplot1 }\OperatorTok{+}\StringTok{ }\KeywordTok{geom_point}\NormalTok{()}
\NormalTok{lolliplot2 }\OperatorTok{+}\StringTok{ }\KeywordTok{scale_x_discrete}\NormalTok{(}\DataTypeTok{limits=}\NormalTok{tableFreqO}\OperatorTok{$}\NormalTok{locale) }\CommentTok{# key element}
\end{Highlighting}
\end{Shaded}

Maybe add some color:

\begin{Shaded}
\begin{Highlighting}[]
\CommentTok{# a new column for color}
\NormalTok{tableFreqO}\OperatorTok{$}\NormalTok{PositiveGap=}\KeywordTok{ifelse}\NormalTok{(tableFreqO}\OperatorTok{$}\NormalTok{gap}\OperatorTok{>}\DecValTok{0}\NormalTok{,T,F)}

\CommentTok{# add new aesthetics 'color'}
\NormalTok{base =}\StringTok{ }\KeywordTok{ggplot}\NormalTok{(tableFreqO, }\KeywordTok{aes}\NormalTok{(locale,gap,}
                              \DataTypeTok{color=}\NormalTok{PositiveGap)) }\CommentTok{#change}
\NormalTok{lolliplot1=base }\OperatorTok{+}\StringTok{ }\KeywordTok{geom_segment}\NormalTok{(}\KeywordTok{aes}\NormalTok{(}\DataTypeTok{y =} \DecValTok{0}\NormalTok{, }
                                   \DataTypeTok{x =}\NormalTok{ locale, }
                                   \DataTypeTok{yend =}\NormalTok{ gap, }
                                   \DataTypeTok{xend =}\NormalTok{ locale), }\DataTypeTok{color =} \StringTok{"gray"}\NormalTok{) }
\NormalTok{lolliplot2 =}\StringTok{ }\NormalTok{lolliplot1 }\OperatorTok{+}\StringTok{ }\KeywordTok{geom_point}\NormalTok{()}
\NormalTok{lolliplot2 }\OperatorTok{+}\StringTok{ }\KeywordTok{scale_x_discrete}\NormalTok{(}\DataTypeTok{limits=}\NormalTok{tableFreqO}\OperatorTok{$}\NormalTok{locale) }\CommentTok{# key element}
\end{Highlighting}
\end{Shaded}

Maybe add some extra info:

\begin{Shaded}
\begin{Highlighting}[]
\CommentTok{# a new column for color}
\NormalTok{tableFreqO}\OperatorTok{$}\NormalTok{PositiveGap=}\KeywordTok{ifelse}\NormalTok{(tableFreqO}\OperatorTok{$}\NormalTok{gap}\OperatorTok{>}\DecValTok{0}\NormalTok{,T,F)}

\NormalTok{base =}\StringTok{ }\KeywordTok{ggplot}\NormalTok{(tableFreqO, }\KeywordTok{aes}\NormalTok{(locale,gap,}\DataTypeTok{color=}\NormalTok{PositiveGap,}
                              \DataTypeTok{label =} \KeywordTok{round}\NormalTok{(gap,}\DecValTok{3}\NormalTok{))) }\CommentTok{#  change}
\NormalTok{lolliplot1=base }\OperatorTok{+}\StringTok{ }\KeywordTok{geom_segment}\NormalTok{(}\KeywordTok{aes}\NormalTok{(}\DataTypeTok{y =} \DecValTok{0}\NormalTok{, }
                                   \DataTypeTok{x =}\NormalTok{ locale, }
                                   \DataTypeTok{yend =}\NormalTok{ gap, }
                                   \DataTypeTok{xend =}\NormalTok{ locale), }\DataTypeTok{color =} \StringTok{"gray"}\NormalTok{) }
\NormalTok{lolliplot2=lolliplot1 }\OperatorTok{+}\StringTok{ }\KeywordTok{geom_point}\NormalTok{() }
\NormalTok{lolliplot3=}\StringTok{ }\NormalTok{lolliplot2 }\OperatorTok{+}\StringTok{ }\KeywordTok{scale_x_discrete}\NormalTok{(}\DataTypeTok{limits=}\NormalTok{tableFreqO}\OperatorTok{$}\NormalTok{locale) }
\CommentTok{# annotating and moving the text on the horizontal}
\NormalTok{lolliplot3 }\OperatorTok{+}\StringTok{ }\KeywordTok{geom_text}\NormalTok{(}\DataTypeTok{nudge_x=}\FloatTok{0.3}\NormalTok{) }
\end{Highlighting}
\end{Shaded}

You can avoid the overlaping symbols in the legend by using:

\begin{Shaded}
\begin{Highlighting}[]
\NormalTok{lolliplot3 }\OperatorTok{+}\StringTok{ }\KeywordTok{geom_text}\NormalTok{(}\DataTypeTok{nudge_x=}\FloatTok{0.3}\NormalTok{,}\DataTypeTok{show.legend =} \OtherTok{FALSE}\NormalTok{) }
\end{Highlighting}
\end{Shaded}

{ Exercise:Complete adding the elements missing in the last plot. }

\protect\hyperlink{part1}{Go to table of contents.}

\paragraph{\texorpdfstring{{ 2. Visualization for ordinal
scales}}{ 2. Visualization for ordinal scales}}\label{visualization-for-ordinal-scales}

For this section, we will use the variable that tells us the highest
grade offered in a school. A simple exploration gives:

\begin{Shaded}
\begin{Highlighting}[]
\KeywordTok{table}\NormalTok{(eduwa}\OperatorTok{$}\NormalTok{High.Grade,}\DataTypeTok{exclude =} \StringTok{'nothing'}\NormalTok{)}
\end{Highlighting}
\end{Shaded}

Being a categorical variable, the default option is again the bar plot.
So let's prepare the frequency table as a data frame:

\begin{Shaded}
\begin{Highlighting}[]
\NormalTok{frqTabO=}\KeywordTok{as.data.frame}\NormalTok{(}\KeywordTok{prop.table}\NormalTok{(}\KeywordTok{table}\NormalTok{(eduwa}\OperatorTok{$}\NormalTok{High.Grade)))}
\KeywordTok{names}\NormalTok{(frqTabO)=}\KeywordTok{c}\NormalTok{(}\StringTok{'grade'}\NormalTok{,}\StringTok{'pct'}\NormalTok{)}
\NormalTok{frqTabO}
\end{Highlighting}
\end{Shaded}

Now, we can use ggplot:

\begin{Shaded}
\begin{Highlighting}[]
\NormalTok{base =}\StringTok{ }\KeywordTok{ggplot}\NormalTok{(frqTabO,}\KeywordTok{aes}\NormalTok{(}\DataTypeTok{x=}\NormalTok{grade,}\DataTypeTok{y=}\NormalTok{pct))}
\NormalTok{base }\OperatorTok{+}\StringTok{ }\KeywordTok{geom_bar}\NormalTok{(}\DataTypeTok{stat =} \StringTok{'identity'}\NormalTok{) }
\end{Highlighting}
\end{Shaded}

The x-values in this variable have \textbf{order}. That is, there is an
increasing level in the values. Whenever we have an ordering, besides
\emph{concentration} we can visualize \textbf{symmetry}: if there is
bias towards lower or higher values.

Bar plots help you see concentration and symmetry, but we have an
alternative way to clearly detect symmetry, via \textbf{boxplots}:

\begin{Shaded}
\begin{Highlighting}[]
\CommentTok{# boxplots do not use frequency tables}

\CommentTok{# as.numeric produces turns levels of the factor into numbers}
\NormalTok{box1 =}\StringTok{ }\KeywordTok{ggplot}\NormalTok{(eduwa, }\KeywordTok{aes}\NormalTok{(}\DataTypeTok{y=}\KeywordTok{as.numeric}\NormalTok{(High.Grade))) }
\NormalTok{box1 =}\StringTok{ }\NormalTok{box1 }\OperatorTok{+}\StringTok{ }\KeywordTok{geom_boxplot}\NormalTok{() }\OperatorTok{+}\StringTok{ }\KeywordTok{coord_flip}\NormalTok{() }\CommentTok{# to show it horizontally}

\NormalTok{box1}
\end{Highlighting}
\end{Shaded}

You have symmetry when the distance of those whiskers to the box is the
same, and when the thick line is in the middle of the box. You can see
that the values show a negative asymmetry (tail to the left).

Box plots expect a numeric value as an input, but we have an ordered
categorical, so we used the \emph{as.numeric()} function. However, that
eliminated the levels we saw in the previous bar plot; we can put the
levels back in our plot:

\begin{Shaded}
\begin{Highlighting}[]
\CommentTok{# the labels use the original ordinal levels}
\NormalTok{ordLabels=}\StringTok{ }\KeywordTok{levels}\NormalTok{(eduwa}\OperatorTok{$}\NormalTok{High.Grade)}

\NormalTok{box2 =}\StringTok{ }\NormalTok{box1 }\OperatorTok{+}\StringTok{ }\KeywordTok{scale_y_continuous}\NormalTok{(}\DataTypeTok{labels=}\NormalTok{ordLabels,}\DataTypeTok{breaks=}\DecValTok{1}\OperatorTok{:}\DecValTok{15}\NormalTok{)}
\NormalTok{box2}
\end{Highlighting}
\end{Shaded}

Box plots have important statistical information. The beginning and the
ending of the box indicates the first (q1) and the third quantile (q75);
and the thicker line in the middle represents the median. All those
values are clearly visible, but we can retrieve the values like this:

\begin{Shaded}
\begin{Highlighting}[]
\CommentTok{#get positions}
\CommentTok{# using 'ggplot_build'}
\NormalTok{pos_q1=}\StringTok{     }\KeywordTok{ggplot_build}\NormalTok{(box2)}\OperatorTok{$}\NormalTok{data[[}\DecValTok{1}\NormalTok{]]}\OperatorTok{$}\NormalTok{lower}
\NormalTok{pos_median=}\StringTok{ }\KeywordTok{ggplot_build}\NormalTok{(box2)}\OperatorTok{$}\NormalTok{data[[}\DecValTok{1}\NormalTok{]]}\OperatorTok{$}\NormalTok{middle}
\NormalTok{pos_q3=}\StringTok{     }\KeywordTok{ggplot_build}\NormalTok{(box2)}\OperatorTok{$}\NormalTok{data[[}\DecValTok{1}\NormalTok{]]}\OperatorTok{$}\NormalTok{upper}

\CommentTok{# using}
\KeywordTok{levels}\NormalTok{(eduwa}\OperatorTok{$}\NormalTok{High.Grade)[}\KeywordTok{c}\NormalTok{(pos_q1,pos_median,pos_q3)]}
\end{Highlighting}
\end{Shaded}

From the information retrieved, we know:

\begin{itemize}
\tightlist
\item
  25\% of the public Schools offer at most 5th GRADE.
\item
  50\% of the public Schools offer at most 8th GRADE.
\item
  75\% of the public Schools offer at most 12th GRADE. Also, 25\% of the
  schools offer at least 12th grade.
\end{itemize}

We can find these results with a \emph{detailed} frequency table; that
is, instead of using the command \emph{table} as we did before, we could
try a more advanced function:

\begin{Shaded}
\begin{Highlighting}[]
\KeywordTok{library}\NormalTok{(summarytools)}
\KeywordTok{freq}\NormalTok{(eduwa}\OperatorTok{$}\NormalTok{High.Grade,}\DataTypeTok{style =} \StringTok{'rmarkdown'}\NormalTok{)}
\end{Highlighting}
\end{Shaded}

{ Exercise: Make sure our box plot follows the same design approach and
include all the elements as in the bar plot for nominal data. }

\protect\hyperlink{part1}{Go to table of contents.}

\subsubsection{Counting}\label{counting}

Counting expresses numerical values. They could be represented with bar
plots if their frequency table had few different values. For example,
the variable \emph{Reduced.Lunch} informs how many kids there are in
each school that have that lunch for a reduced price.

\begin{Shaded}
\begin{Highlighting}[]
\CommentTok{# how many unique values}
\KeywordTok{length}\NormalTok{(}\KeywordTok{unique}\NormalTok{(eduwa}\OperatorTok{$}\NormalTok{Reduced.Lunch))}
\end{Highlighting}
\end{Shaded}

There are too many different values. Then, the bar plot is not a good
idea (and neither a frequency table):

\begin{Shaded}
\begin{Highlighting}[]
\KeywordTok{barplot}\NormalTok{(}\KeywordTok{table}\NormalTok{(eduwa}\OperatorTok{$}\NormalTok{Reduced.Lunch),}\DataTypeTok{las=}\DecValTok{2}\NormalTok{,}\DataTypeTok{cex.names =} \FloatTok{0.3}\NormalTok{,}
        \DataTypeTok{main=}\StringTok{'bad idea'}\NormalTok{)}
\end{Highlighting}
\end{Shaded}

On the other hand, when we have a numerical variable, there are more
statistical values that help understand its behavior:

\begin{Shaded}
\begin{Highlighting}[]
\CommentTok{# median close to mean?}
\CommentTok{# median and mean far from max or min?}
\CommentTok{# q1 distance to min is similar ti q3 distance to max?}
\CommentTok{# how many missing?}

\KeywordTok{summary}\NormalTok{(eduwa}\OperatorTok{$}\NormalTok{Reduced.Lunch)}
\end{Highlighting}
\end{Shaded}

The bar plot produces a bar for each unique value in the data, counting
how many times this value appeared. Now, we have many values, so we need
to organize the data into \emph{intervals}. The \textbf{histogram} is
the basic plot when intervals are needed, you can use the basic
function:

\begin{Shaded}
\begin{Highlighting}[]
\NormalTok{eduwa3=eduwa[}\KeywordTok{complete.cases}\NormalTok{(eduwa}\OperatorTok{$}\NormalTok{Reduced.Lunch),]}
\NormalTok{dataHist=}\KeywordTok{hist}\NormalTok{(eduwa3}\OperatorTok{$}\NormalTok{Reduced.Lunch) }\CommentTok{#saving info in dataHist}
\end{Highlighting}
\end{Shaded}

The width of each \textbf{bin} (bar) represents an interval of values,
while its height the frequency. The histogram shows an asymmetric shape,
where the bin with lowest values of the variable (between 0 and 20) are
the most common (above 1000).

Of course, ggplot has a version of histograms:

\begin{Shaded}
\begin{Highlighting}[]
\NormalTok{base=}\StringTok{ }\KeywordTok{ggplot}\NormalTok{(eduwa3,}\KeywordTok{aes}\NormalTok{(}\DataTypeTok{x =}\NormalTok{ Reduced.Lunch))  }
\NormalTok{h1=}\StringTok{ }\NormalTok{base }\OperatorTok{+}\StringTok{ }\KeywordTok{geom_histogram}\NormalTok{()}
\NormalTok{h1 }
\end{Highlighting}
\end{Shaded}

Notice that you do not get the same plot. Let's see the info from the
basic function:

\begin{Shaded}
\begin{Highlighting}[]
\NormalTok{dataHist}
\end{Highlighting}
\end{Shaded}

And now see the info that was used in ggplot:

\begin{Shaded}
\begin{Highlighting}[]
\KeywordTok{ggplot_build}\NormalTok{(h1)}\OperatorTok{$}\NormalTok{data[[}\DecValTok{1}\NormalTok{]]}
\end{Highlighting}
\end{Shaded}

The first `x' was 0 in ggplot, while it was 10 (in \emph{\$mids}) in the
base graphic; from there on everything changed. And not only that, you
have 16 bins in the base graphic, while you got 30 in ggplot.

Of course, you can alter that in both alternatives.

Below, you can see a version where both plots are the same:

\begin{Shaded}
\begin{Highlighting}[]
\CommentTok{#ggplot}
\NormalTok{base=}\StringTok{ }\KeywordTok{ggplot}\NormalTok{(eduwa3,}\KeywordTok{aes}\NormalTok{(}\DataTypeTok{x =}\NormalTok{ Reduced.Lunch))  }
\NormalTok{h1=}\StringTok{ }\NormalTok{base }\OperatorTok{+}\StringTok{ }\KeywordTok{geom_histogram}\NormalTok{(}\DataTypeTok{binwidth =} \DecValTok{20}\NormalTok{,}\DataTypeTok{boundary=}\DecValTok{0}\NormalTok{) }\CommentTok{#changing width}
\NormalTok{h1=}\StringTok{ }\NormalTok{h1 }\OperatorTok{+}\StringTok{ }\KeywordTok{stat_bin}\NormalTok{(}\DataTypeTok{binwidth =} \DecValTok{20}\NormalTok{, }\KeywordTok{aes}\NormalTok{(}\DataTypeTok{label=}\NormalTok{..count..), }
                  \DataTypeTok{geom =} \StringTok{"text"}\NormalTok{,}\DataTypeTok{boundary =} \DecValTok{0}\NormalTok{,}\DataTypeTok{vjust=}\OperatorTok{-}\FloatTok{0.5}\NormalTok{)}
\NormalTok{h1}
\end{Highlighting}
\end{Shaded}

\begin{Shaded}
\begin{Highlighting}[]
\CommentTok{# base}
\KeywordTok{hist}\NormalTok{(eduwa3}\OperatorTok{$}\NormalTok{Reduced.Lunch,}\DataTypeTok{labels =}\NormalTok{ T,}\DataTypeTok{xlab=}\StringTok{"Reduced Lunch"}\NormalTok{)}
\end{Highlighting}
\end{Shaded}

Of course, you can make it a litle better:

\begin{Shaded}
\begin{Highlighting}[]
\KeywordTok{hist}\NormalTok{(eduwa3}\OperatorTok{$}\NormalTok{Reduced.Lunch,}\DataTypeTok{labels =}\NormalTok{ T,}\DataTypeTok{xlab=}\StringTok{"Reduced Lunch"}\NormalTok{, }\DataTypeTok{xaxt=}\StringTok{"n"}\NormalTok{) }
\KeywordTok{axis}\NormalTok{(}\DataTypeTok{side=}\DecValTok{1}\NormalTok{, }\DataTypeTok{at=}\NormalTok{dataHist}\OperatorTok{$}\NormalTok{breaks) }\CommentTok{# showing axis labels better}
\end{Highlighting}
\end{Shaded}

As mentioned before, we are plotting intervals, so the accompanying
table can be built. For that, we first create the intervals into another
variable:

\begin{Shaded}
\begin{Highlighting}[]
\NormalTok{eduwa3}\OperatorTok{$}\NormalTok{redLunchOrd=}\KeywordTok{cut}\NormalTok{(eduwa3}\OperatorTok{$}\NormalTok{Reduced.Lunch,}
                       \DataTypeTok{breaks =}\NormalTok{ dataHist}\OperatorTok{$}\NormalTok{breaks,}
                       \DataTypeTok{include.lowest =}\NormalTok{ T,}
                       \DataTypeTok{ordered_result =}\NormalTok{ T)}
\end{Highlighting}
\end{Shaded}

And, as before, we use the \emph{freq} function:

\begin{Shaded}
\begin{Highlighting}[]
\CommentTok{# no need to show count of NAs:}
\KeywordTok{freq}\NormalTok{(eduwa3}\OperatorTok{$}\NormalTok{redLunchOrd,}\DataTypeTok{style =} \StringTok{'rmarkdown'}\NormalTok{,}\DataTypeTok{report.nas =}\NormalTok{ F)}
\end{Highlighting}
\end{Shaded}

{ Exercise: Make a histogram for the variable FREE LUNCH, and make sure
it has all the right elements, and get rid of unnecessary elements. }

\protect\hyperlink{part1}{Go to table of contents.}

\subsubsection{Measurement}\label{measurement}

A simplistic idea of measurement tells you the times a particular unit
is present in the unit of analysis; which allows for the presence of
decimal places. There are variables that can have negative values.

Let's analyze the variable \emph{Student.Teacher.Ratio}, but organized
by county:

\begin{Shaded}
\begin{Highlighting}[]
\CommentTok{# tapply(variable,group,functionToApply)}
\KeywordTok{tapply}\NormalTok{(eduwa}\OperatorTok{$}\NormalTok{Student.Teacher.Ratio, eduwa}\OperatorTok{$}\NormalTok{County, mean)}
\end{Highlighting}
\end{Shaded}

Above, I tried to compute the mean for each county, but the function
\emph{mean()} outputs a missing value (\emph{NA}) as the result when
there is one \emph{NA} in the column:

\begin{Shaded}
\begin{Highlighting}[]
\CommentTok{# strategy 1: remove missing before computing function: na.rm=T}
\KeywordTok{tapply}\NormalTok{(eduwa}\OperatorTok{$}\NormalTok{Student.Teacher.Ratio, eduwa}\OperatorTok{$}\NormalTok{County, mean,}\DataTypeTok{na.rm=}\NormalTok{T)}
\end{Highlighting}
\end{Shaded}

Of course, you can clean first:

\begin{Shaded}
\begin{Highlighting}[]
\CommentTok{# strategy 2: }
\NormalTok{eduwa4=eduwa[}\KeywordTok{complete.cases}\NormalTok{(eduwa}\OperatorTok{$}\NormalTok{Student.Teacher.Ratio),]}

\KeywordTok{tapply}\NormalTok{(eduwa4}\OperatorTok{$}\NormalTok{Student.Teacher.Ratio, }
\NormalTok{       eduwa4}\OperatorTok{$}\NormalTok{County, }
\NormalTok{       mean)}
\end{Highlighting}
\end{Shaded}

Great!

Now let me plot a histogram of those means:

\begin{Shaded}
\begin{Highlighting}[]
\CommentTok{# keeping strategy 2: }
\NormalTok{meanValues=}\KeywordTok{tapply}\NormalTok{(eduwa4}\OperatorTok{$}\NormalTok{Student.Teacher.Ratio, }
\NormalTok{                  eduwa4}\OperatorTok{$}\NormalTok{County, }
\NormalTok{                  mean)}
\KeywordTok{hist}\NormalTok{(meanValues)}
\end{Highlighting}
\end{Shaded}

Let's compute some statistics:

\begin{Shaded}
\begin{Highlighting}[]
\KeywordTok{summary}\NormalTok{(meanValues)}
\end{Highlighting}
\end{Shaded}

You can use that info, for example, to plot the mean as a reference
line:

\begin{Shaded}
\begin{Highlighting}[]
\CommentTok{#reference line}
\KeywordTok{hist}\NormalTok{(meanValues)}
\KeywordTok{abline}\NormalTok{(}\DataTypeTok{v=}\KeywordTok{mean}\NormalTok{(meanValues),}\DataTypeTok{lty=}\DecValTok{3}\NormalTok{,}\DataTypeTok{lwd=}\DecValTok{3}\NormalTok{,}\DataTypeTok{col=}\StringTok{'blue'}\NormalTok{)}
\end{Highlighting}
\end{Shaded}

Measurements are continuous values, then a \textbf{density} plot is more
appealing to its nature:

\begin{Shaded}
\begin{Highlighting}[]
\NormalTok{mvDense=}\KeywordTok{density}\NormalTok{(meanValues)}

\KeywordTok{plot}\NormalTok{(mvDense,}\DataTypeTok{main=}\StringTok{"Title"}\NormalTok{,}\DataTypeTok{col=}\StringTok{'black'}\NormalTok{,}\DataTypeTok{xlab=}\OtherTok{NA}\NormalTok{)}

\KeywordTok{abline}\NormalTok{(}\DataTypeTok{v=}\KeywordTok{mean}\NormalTok{(meanValues),}\DataTypeTok{lty=}\DecValTok{3}\NormalTok{,}\DataTypeTok{lwd=}\DecValTok{3}\NormalTok{,}\DataTypeTok{col=}\StringTok{'blue'}\NormalTok{) }\CommentTok{#mean}
\KeywordTok{abline}\NormalTok{(}\DataTypeTok{v=}\KeywordTok{median}\NormalTok{(meanValues),}\DataTypeTok{lty=}\DecValTok{3}\NormalTok{,}\DataTypeTok{lwd=}\DecValTok{3}\NormalTok{,}\DataTypeTok{col=}\StringTok{'red'}\NormalTok{)}\CommentTok{#median}
\KeywordTok{legend}\NormalTok{(}\DataTypeTok{x=}\StringTok{"right"}\NormalTok{,}
       \DataTypeTok{legend=}\KeywordTok{c}\NormalTok{(}\StringTok{'mean'}\NormalTok{,}\StringTok{'median'}\NormalTok{),}
       \DataTypeTok{fill =} \KeywordTok{c}\NormalTok{(}\StringTok{'blue'}\NormalTok{,}\StringTok{'red'}\NormalTok{),}\DataTypeTok{bty =} \StringTok{'n'}\NormalTok{) }\CommentTok{#no box in the legend}
\end{Highlighting}
\end{Shaded}

A box plot is always welcome, specially considering that it does not
need reference lines. Take a look:

\begin{Shaded}
\begin{Highlighting}[]
\NormalTok{bp=}\KeywordTok{boxplot}\NormalTok{(meanValues,}\DataTypeTok{horizontal =}\NormalTok{ T,}\DataTypeTok{ylim=}\KeywordTok{c}\NormalTok{(}\DecValTok{5}\NormalTok{,}\DecValTok{30}\NormalTok{))}
\end{Highlighting}
\end{Shaded}

Our plots for the mean values have a more symmetrical shape. This
happens when you get mean values of groups, showing a tendency towards a
bell-shaped distribution, which is ideally known as the \emph{Gauss} or
\emph{Normal} distribution.

Notice also that boxplots serve to detect \textbf{atypical} values
(outliers), which I saved in \emph{bp}:

\begin{Shaded}
\begin{Highlighting}[]
\NormalTok{bp}\OperatorTok{$}\NormalTok{out}
\end{Highlighting}
\end{Shaded}

We could annotate the boxplot like this:

\begin{Shaded}
\begin{Highlighting}[]
\KeywordTok{boxplot}\NormalTok{(meanValues,}\DataTypeTok{horizontal =}\NormalTok{ T,}\DataTypeTok{ylim=}\KeywordTok{c}\NormalTok{(}\DecValTok{5}\NormalTok{,}\DecValTok{30}\NormalTok{))}
\KeywordTok{text}\NormalTok{(}\DataTypeTok{x=} \DecValTok{10}\NormalTok{, }\DataTypeTok{y=} \FloatTok{0.8}\NormalTok{, }\DataTypeTok{labels=} \StringTok{"Outliers are:"}\NormalTok{,}\DataTypeTok{col=}\StringTok{'gray'}\NormalTok{)}
\KeywordTok{text}\NormalTok{(}\DataTypeTok{x=} \DecValTok{10}\NormalTok{, }\DataTypeTok{y=} \FloatTok{0.75}\NormalTok{, }
     \DataTypeTok{labels=} \KeywordTok{paste}\NormalTok{(}\KeywordTok{names}\NormalTok{(bp}\OperatorTok{$}\NormalTok{out)[}\DecValTok{1}\NormalTok{], }\StringTok{'and'}\NormalTok{, }\KeywordTok{names}\NormalTok{(bp}\OperatorTok{$}\NormalTok{out)[}\DecValTok{2}\NormalTok{]),}
     \DataTypeTok{col=}\StringTok{'gray'}\NormalTok{)}
\end{Highlighting}
\end{Shaded}

In general, measurements and counts are prone to have outliers. It is
not common to speak about outliers in categorical data since they have
few levels; however, if they had many levels, we could find outliers if
the variable is ordinal.

From what I said above, the subjective side of finding outliers lies in
the decision of \textbf{what is normal}. In the case of the boxplot, the
decision has been to accept as normal the values that have a
\emph{prudent distance} from the first or last quartile. This distance
is 1.5 times the difference between the quartiles (a.k.a. Interquartle
Range or \textbf{IQR}). Then, if a outlier is found, the whisker ends in
a position different than the actual minimum or maximal value of the
data.

{ Exercise: Do some research and make a histogram and a density plot
using ggplot for the variable we just used above. }

\begin{center}\rule{0.5\linewidth}{\linethickness}\end{center}

\protect\hyperlink{part1}{Go to table of contents.}

\href{https://evansdatascience.github.io/VisualAnalytics/}{Back to
course schedule menu}


\end{document}
